\subsection{Testing uniformity}

\todo{DECIDE WHICH TEST TO USE, BETWEEN: MINIMUM DISTANCE, BINNED SUMS, SIMPLE NUMBER FREQUENCIES}

\todo{Decide whether we should specify the multivariate hypergeometric assumption.
Decide whether this assumption would change the degrees of freedom (at the moment, I think it wouldn't).}

We are interested in whether each of the lottery numbers appear with equal probability.
However, each sample of 6 balls from the lottery bowl is sampled without replacement. 
Therefore, within a single lottery drawing, the balls are *not* independent. 

Each lottery day is independent. So we can 

We conducted the following hypothesis test.

$H_0$: each lottery number is drawn with equal probability.

$H_A$: the lottery numbers are drawn with unequal probabilities: some numbers are more likely to appear than others.

We used Pearson's $\chi^2$ test, a commonly recommended test for the probabilities of observing categorical data. \todo{Motivate the choice of this particular test.}

% In particular, we model the lottery drawing with a multivariate hypergeometric distribution. \todo{Give a more detailed explanation of this distribution, and argue that it's appropriate.}

% The multivariate hypergeometric distribution models the following situation: given an urn filled with balls of 
% different colors, sample $n$ balls without replacement from the urn. How many balls of each color are in the sample?

% This is analogous to the lottery, when we consider the lottery bowl as the urn and the numbers on the balls as the colors.

% The assumption that each color is equally likely to be drawn is built into this distribution: only the number of balls of each color affects the probability of that color being drawn.
% Therefore, the multivariate hypergeometric distribution is an appropriate choice to test the null hypothesis.

% More formally, the hypothesis test can be stated in the following way.

% $H_0$: the lottery drawing follows a multivariate hypergeometric distribution with parameters: \todo{State the parameters}.

% $H_A$: the lottery drawing does not follow a multivariate hypergeometric distribution.

Under $H_0$, we compute the following expected frequencies of each lottery number: \todo{Create a table with the expected frequencies. They will all be the same, so perhaps this can just be stated in a sentence.}

The $\chi^2$ statistic is a function of how much our actual observed frequencies deviate from these expected frequencies. 
Larger deviations result in higher values of the $\chi^2$ statistic and therefore lower p-values.

\todo{Justify the choice of degrees of freedom.}

\todo{Decide whether we should spell out the computation of the p-value, as if we computed it manually.}

\todo{Report the value of the $\chi^2$ statistic and the p-value.}

\subsection{Diehard tests}

For more information about the Diehard battery of tests we refer the reader to the original paper~\cite{currentRNG}.
