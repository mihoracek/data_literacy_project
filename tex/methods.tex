\subsection{Testing uniformity}

\todo{QUESTION FOR MICHAEL: which test should we use, between: MINIMUM DISTANCE, BINNED SUMS, SIMPLE NUMBER FREQUENCIES.
Is it okay if the analysis is essentially: replicating the methods in a paper, but on a new dataset?}

We conducted the following hypothesis test.

$H_0$: each lottery number is drawn with equal probability.

$H_A$: the lottery numbers are drawn with unequal probabilities: some numbers are more likely to appear than others.

We used Pearson's $\chi^2$ test, a commonly recommended test for the probabilities of observing categorical data. \todo{Motivate the choice of this particular test.}

\subsubsection{Description of minimum-distance statistic.}

To create the frequency tables for the $\chi^2 test$, we used the minimum distance statistic $d$ used by Drakakis et al.

\todo{write the definition from Drakakis et al.}

This statistic is useful for detecting human tampering because it is known that humans usually do a
poor job of imitating the random choices that occur under the null hypothesis. Boland and Pawitan showed
that when humans are asked to sample $m$ integers from the set ${1,...,n}$, the value
of this statistic is greater on average than it is under uniform sampling. In particular,
they observed that the true probability of $d = 1$ is greater than $0.5$, a result that 
is highly unintuitive to humans: humans tend to underestimate how likely it is that
two consecutive numbers are picked. In particular, Boland and Pawitan
computed a $\chi^2$ goodness-of-fit statistic of the human-produced $d$ against the uniformly produced $d$
, which if they had conducted a hypothesis test, would have yielded a 
p-value of < 0.0000001. 

Therefore, a low p-value could be consistent with human tampering, especially if we
observe an unusually small number of lottery drawings where $d = 1$, as Drakakis et al.
did in the French lottery. 

\subsubsection{Description of the test statistic.}

\todo{Explain the $\chi^2$ goodness-of-fit test. 
Include a table of the expected vs. observed frequencies.
Report the results.}

\todo{I would put the following paragraph into the distribution testing section, as its not relevant to my diehard part.}

To prepare the German Lotto dataset for the hypothesis test, we computed $d$, the minimum distance between winning numbers, for each 
lottery day. We counted the frequencies of each value of $d$, and combined the counts of the 
two largest possible values, $7$ and $8$. This is the same preparation that was performed in the paper,
and we did it to satisfy the common rule of thumb for usage of the 
$\chi^2$ goodness-of-fit test: the expected frequency in each bin must be $\geq 5$.

\begin{table}

    \caption{Frequencies of $d$ statistic for German Lotto}
    \centering
    \begin{tabular}[t]{lrr}
    \toprule
    d & Expected frequency & Actual frequency\\
    \midrule
    1 & 2310.595965 & 2383\\
    2 & 1266.759926 & 1247\\
    3 & 639.888057 & 615\\
    4 & 290.255446 & 273\\
    5 & 113.589766 & 105\\
    \addlinespace
    6 & 35.857667 & 35\\
    7 and 8 & 9.025145 & 8\\
    \bottomrule
    \end{tabular}
    \end{table}

% \todo{Decide whether we should specify the multivariate hypergeometric assumption.
% Decide whether this assumption would change the degrees of freedom (at the moment, I think it wouldn't).}

% We are interested in whether each of the lottery numbers appear with equal probability.
% However, each sample of 6 balls from the lottery bowl is sampled without replacement. 
% Therefore, within a single lottery drawing, the balls are *not* independent. 

% In particular, we model the lottery drawing with a multivariate hypergeometric distribution. \todo{Give a more detailed explanation of this distribution, and argue that it's appropriate.}

% The multivariate hypergeometric distribution models the following situation: given an urn filled with balls of 
% different colors, sample $n$ balls without replacement from the urn. How many balls of each color are in the sample?

% This is analogous to the lottery, when we consider the lottery bowl as the urn and the numbers on the balls as the colors.

% The assumption that each color is equally likely to be drawn is built into this distribution: only the number of balls of each color affects the probability of that color being drawn.
% Therefore, the multivariate hypergeometric distribution is an appropriate choice to test the null hypothesis.

% More formally, the hypothesis test can be stated in the following way.

% $H_0$: the lottery drawing follows a multivariate hypergeometric distribution with parameters: \todo{State the parameters}.

% $H_A$: the lottery drawing does not follow a multivariate hypergeometric distribution.

% Under $H_0$, we compute the following expected frequencies of each lottery number: \todo{Create a table with the expected frequencies.}

The $\chi^2$ statistic is a function of how much our actual observed frequencies deviate from these expected frequencies. 
Larger deviations result in higher values of the $\chi^2$ statistic and therefore lower p-values.

\todo{Justify the choice of degrees of freedom.}

\todo{Decide whether we should spell out the computation of the p-value, as if we computed it manually.}

\todo{Report the value of the $\chi^2$ statistic and the p-value.}

\subsection{Diehard tests}

For more information about the Diehard battery of tests we refer the reader to the original paper~\cite{currentRNG}.
