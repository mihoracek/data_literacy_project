Section \ref{sec:methods} investigates numbers drawn from the German lottery Lotto, from 1955 onwards. 
We have chosen this dataset because it is relevant to us and readers of this analysis as current residents of Germany, 
and it contains almost 70 years of data, which we believ is enough data to perform a meaningful analysis.

Furthermore, Drakakis, Taylor, and Rickard did not apply their tests to this data, so 
performing the test on the German Lotto data is a meaningful new result.

The dataset has been compiled by Johannes Friedrich, 
a software developer who has made the data publicly available via 
\href{https://github.com/JohannesFriedrich/LottoNumberArchive}{his Github repository}.

\todo{Perform verification of the correctness of the dataset by sampling from it and manually inspecting the data.}

\todo{I would put the following paragraph into the distribution testing section, as its not relevant to my diehard part.}

To prepare the dataset for the hypothesis test, we computed $d$, the minimum distance between winning numbers, for each 
lottery day. We counted the frequencies of each value of $d$, and combined the counts of the 
two largest possible values, $7$ and $8$. This is the same preparation that was performed in the paper,
and we did it to satisfy the common rule of thumb for usage of the 
$\chi^2$ goodness-of-fit test: the expected frequency in each bin must be $\geq 5$.

\todo{Is there a dataset for a rigged lottery?} \href{https://notebook.community/JesseScott/Lotto649/lotto}{This} looks sketchy as hell.

However even 70 years of Lotto numbers is not sufficient to produce enough data for the diehard tests. Their author recommends
using 10 to 12 MB of random bits, which is substantially more than 28.4 KiB of Lotto numbers. Therefore we add winning numbers
from other lotteries to our dataset. By data volume, our data is dominated by the New York Quick Draw lottery, because it has
been drawn every 4 minutes for the last decade, adding up to to 25,663,100 numbers at our collection of the dataset.
Washington DC's Keno is quite similar, reaching 2,299,563 numbers over 3 years of existence. Our dataset is completed by
the "other" category, consisting of 1,917,890 winning numbers combined from lotteries drawn less often worldwide.

\begin{figure}
    \centering
    \includegraphics[width=\textwidth]{dataset_composition.pdf}
    \caption{Origin of numbers making up our dataset.}
    \label{fig:dataset}
\end{figure}

The "other" dataset comes from European or english-speaking countries. The reason is twofold - we have been looking
for them with English search queries and these nations are more likely to subscribe to ideas like open data and thus
offer such dataset as csv files. The European nationality stands for European transnational lotteries Eurojackpot and Euromillions.

\begin{figure}
    \centering
    \includegraphics[width=\textwidth]{other_composition.pdf}
    \caption{National composition of the "other" numbers.}
\end{figure}

In total, we have gathered winning numbers from 37 different lotteries.
