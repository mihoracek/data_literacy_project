The main dataset that we investigate is the numbers drawn from the German lottery Lotto, from 1955 onwards. 
We have chosen this dataset because it is relevant to us and readers of this analysis as current residents of Germany, 
and it contains almost 70 years of data, which we believ is enough data to perform a meaningful analysis.

Furthermore, Drakakis, Taylor, and Rickard did not apply their tests to this data, so 
performing the test on the German Lotto data is a meaningful new result.

The dataset has been compiled by Johannes Friedrich, 
a software developer who has made the data publicly available via 
\href{https://github.com/JohannesFriedrich/LottoNumberArchive}{his Github repository}.

\todo{Perform verification of the correctness of the dataset by sampling from it and manually inspecting the data.}

To prepare the dataset for the hypothesis test, we computed $d$, the minimum distance between winning numbers, for each 
lottery day. We counted the frequencies of each value of $d$, and combined the counts of the 
two largest possible values, $7$ and $8$. This is the same preparation that was performed in the paper,
and we did it to satisfy the common rule of thumb for usage of the 
$\chi^2$ goodness-of-fit test: the expected frequency in each bin must be $\geq 5$.

% The chief component of our dataset is formed by numbers drawn in German lottery Lotto from 1955 onwards. We have focused on these dataset because of geographical locality
% and historical depth of 70 years. Since its beginning, the rules have undergone slight changes, such as the introduction of "super numbers" in December
% of 1991. Nonetheless, the core principle of the lottery, six numbers between 1 and 49 has not changed once in almost 70 years and thus provides a consistent
% basis for our work.

\todo{Is there a dataset for a rigged lottery?} \href{https://notebook.community/JesseScott/Lotto649/lotto}{This} looks sketchy as hell.

However even 70 years of Lotto numbers is not sufficient to produce enough data for the diehard tests. These require 10 to 12 MiB of random bits, which is
substantially more than 28.4 KiB of Lotto numbers. Therefore we downloaded other lottery datasets and combined them together. In total, our dataset reached
more than 750 000 numbers drawn in 18 different lotteries. These majority of these lotteries come from various english-speaking countries such as USA,
Australia or UK because we have been looking for them with English search queries. Most of the complement is formed by other european nations like Italy,
Czech Republic or Germany.

\todo{Merge datasets?}

The numbers are drawn individually, but their order within a single lottery draw does not matter - but maybe it does for some of our tests?

\todo{Investigate this}
