Lottery is a prominent source of chance outcomes in everyday life. However, we believe it is not a given that lotteries are conducted fairly. This assumption has been tested in prior research ~\cite{Drakakis}, and we believe it is worth testing with new data, especially with the German Lotto data, which is relevant to the authors and readers of this analysis as residents of Germany. Furthermore, even if a lottery has not been tampered with, assessing the randomness of different lotteries is useful to determine if they are truly fair, and to demonstrate that patterns do exist: players might investigate further to find and exploit these patterns for monetary gain.

This report consists of four sections. Section \ref{sec:dataset} describes the dataset collection. In section \ref{sec:distribution_testing}, we choose two lotteries to apply the minimum distance test from~\cite{Drakakis} to. Subsequently, we explore the randomness of lottery through the Diehard battery of tests in section \ref{sec:diehard}. Finally we briefly discuss the limitations of our work in section \ref{sec:conclusion}. Our code and data are available \href{https://github.com/mihoracek/data_literacy_project}{here}.