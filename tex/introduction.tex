Games such as lottery and dice are likely one of the first application of randomness in human culture.
But for the same time organizers have cheated in lotteries, skewing the uniform distribution of drawn numbers expected by common sense for their
own monetary gain and misleading customers.

We begin by describing the data-gathering process in Section \ref{sec:dataset}. Given the enormous demands placed on data volume by the second
part of this paper, a sizable portion of our work involved preparing the input data.

In Section \ref{sec:methods}, we investigate the distribution of answers for the German Lotto lottery drawn from 1955 to the present day.

We continue by reformulating lottery as a process generating random numbers and explore their quality via the Diehard battery of tests~\cite{diehard}
in Section \ref{sec:diehard}. Through their means we attempt to prove whether lottery holds properties such as mutual uncorrelatedness or an absence
of a period of repetition.

This paper is concluded in Section \ref{sec:conclusion} by a brief discussion of the limitations of our work.
