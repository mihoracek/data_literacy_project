We analyzed the uniformity of lottery results in two major ways. We performed a hypothesis test of the distribution of the minimum distance $d$, a statistic that is known to be poorly imitated by humans. Based on the results for the German Lotto and NY Quick Draw, there is no strong evidence for human tampering, but it is unknown how well this analysis may detect other forms of unfairness, and it is unknown how well this would inform a player's choices.

% The Diehard analysis was greatly limited by insufficient amount of data and in some cases unsuitable format in which the data was reported in (ascending-order draws). Out of the three datasets, we can only make a semi-conclusive statement about DC Keno. Here the evidence points towards the validity of the null hypothesis; that is, the randomness and therefore fairness of the lottery.

The Diehard analysis was greatly limited by having too little data, the without-replacement nature of lottery sampling, and some obviously non-random (sorted) data. Despite these concerns, we believe that the performance of DC Keno on the Diehard tests suggests the fairness of this lottery.
