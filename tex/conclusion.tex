\begin{enumerate}
    \item Lottery draws without replacement
    \item Some datasets are sorted in draw order
    \item Reminder: too few numbers for comfort
    \item The Diehard tests are flawed (\href{https://crypto.stackexchange.com/a/397}{Linear Feedback Shift Registers})
\end{enumerate}

We initially wanted to perform a hypothesis test of the distribution of the numbers. 
In particular, we considered performing a test of the goodness-of-fit of 
the discrete uniform distribution from $1$ to $49$ for the observed frequencies
of each Lotto number. However, we discarded this approach because the common goodness-of-fit tests
($\chi^2$, Kolmogorov-Smirnov) require that the events are i.i.d., and the Lotto numbers
are not i.i.d.: in each Lotto drawing, the numbers are sampled without replacement,
so each number drawn on a particular day depends on the numbers that were drawn before
on that day.

Each lottery day is i.i.d. from a multivariate hypergeometric distribution,
so we can apply one of these goodness-of-fit tests when each day contributes a single
event. However, another rule-of-thumb for the $\chi^2$ goodness-of-fit test is that
each category should have an expected count $\geq 5$. Directly counting the frequencies
of each unique combination would not satisfy this rule-of-thumb. To handle this, we could compute
a function of lottery draws and use that to group lottery draws together: for example,
we could compute the sum of the 6 Lotto numbers from a single drawing, and count
the frequencies of the sums being in specific bins. Then, we could safely apply the 
$\chi^2$ goodness-of-fit test. However, it is unknown how well this approach may
detect human tampering. Prior research has shown that humans do a poor job of 
reproducing the distribution of $d$, so we felt that it was more appropriate for 
tampering detection. However, this approach may still be useful for a player of the lottery:
a deviation from the expected distribution of sums may guide a player to choose
a combination of numbers that is more likely than would be expected under a "fair"
null hypothesis.