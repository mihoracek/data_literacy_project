\documentclass{article}

% if you need to pass options to natbib, use, e.g.:
%     \PassOptionsToPackage{numbers, compress}{natbib}
% before loading neurips_2021

% ready for submission
\usepackage[preprint]{neurips_2021}

% to compile a preprint version, e.g., for submission to arXiv, add add the
% [preprint] option:
%     \usepackage[preprint]{neurips_2021}

% to compile a camera-ready version, add the [final] option, e.g.:
%     \usepackage[final]{neurips_2021}

% to avoid loading the natbib package, add option nonatbib:
%    \usepackage[nonatbib]{neurips_2021}

\usepackage[utf8]{inputenc} % allow utf-8 input
\usepackage[T1]{fontenc}    % use 8-bit T1 fonts
\usepackage[colorlinks=true]{hyperref}       % hyperlinks
\usepackage{url}            % simple URL typesetting
\usepackage{booktabs}       % professional-quality tables
\usepackage{amsfonts}       % blackboard math symbols
\usepackage{nicefrac}       % compact symbols for 1/2, etc.
\usepackage{microtype}      % microtypography
\usepackage{xcolor}         % colors
\usepackage{graphicx}       % Extended version of graphics

\newcommand{\todo}[1]{\textcolor{red}{\textbf{#1}}}

\graphicspath{ {./images/} }

\title{Is lottery fair?}

% The \author macro works with any number of authors. There are two commands
% used to separate the names and addresses of multiple authors: \And and \AND.
%
% Using \And between authors leaves it to LaTeX to determine where to break the
% lines. Using \AND forces a line break at that point. So, if LaTeX puts 3 of 4
% authors names on the first line, and the last on the second line, try using
% \AND instead of \And before the third author name.

\author{%
  Michal Horáček\\
  Matrikelnummer 6373382\\
  \texttt{michal.horacek@student.uni-tuebingen.de} \\
  \And
  Carson Zhang\\
  Matrikelnummer 6384481\\
  \texttt{carson.zhang@student.uni-tuebingen.de} \\
}

\begin{document}

\maketitle

\begin{abstract}
    We investigate the uniformity of various lotteries. 
    First, we test the null hypothesis that, for the German Lotto, the distribution of the minimum distance between
    lottery numbers, $d$, is as it would be in a uniformly sampled lottery. The $\chi^2_6$ goodness-of-fit
    test resulted in a p-value of $0.4989$.
    Furthermore, to evaluate the randomness of the random number generators underlying different lotteries,
    we apply the Diehard battery of tests \todo{with the following results}.
\end{abstract}

\section{Introduction}
\label{sec:introduction}
Lottery/Hazard business is big many. Games such as lottery and dice are (probably?) one of the first application of randomness in human culture.
But for the same time organizers have cheated in lotteries, skewing the uniform distribution of drawn numbers expected by common sense for their
own monetary gain. We investigate the distribution of answers for the German Lotto lottery drawn from 1955 to the present day.

In the second part of this paper, we explore the "quality" of random numbers generated by lottery draws via the Diehard battery of tests~\cite{diehard}. These
tests were developed in the 1990's by George Marsaglia and are still in use today to test random number generators~\cite{modernRNG}.


\section{Dataset}
\label{sec:dataset}
The main dataset that we investigate is the numbers drawn from the German lottery Lotto, from 1955 onwards. 
We have chosen this dataset because it is relevant to us and readers of this analysis as current residents of Germany, 
and it contains almost 70 years of data, which we believ is enough data to perform a meaningful analysis.

Furthermore, Drakakis, Taylor, and Rickard did not apply their tests to this data, so 
performing the test on the German Lotto data is a meaningful new result.

The dataset has been compiled by Johannes Friedrich, 
a software developer who has made the data publicly available via 
\href{https://github.com/JohannesFriedrich/LottoNumberArchive}{his Github repository}.

\todo{Perform verification of the correctness of the dataset by sampling from it and manually inspecting the data.}

To prepare the dataset for the hypothesis test, we computed $d$, the minimum distance between winning numbers, for each 
lottery day. We counted the frequencies of each value of $d$, and combined the counts of the 
two largest possible values, $7$ and $8$. This is the same preparation that was performed in the paper,
and we did it to satisfy the common rule of thumb for usage of the 
$\chi^2$ goodness-of-fit test: the expected frequency in each bin must be $\geq 5$.

% The chief component of our dataset is formed by numbers drawn in German lottery Lotto from 1955 onwards. We have focused on these dataset because of geographical locality
% and historical depth of 70 years. Since its beginning, the rules have undergone slight changes, such as the introduction of "super numbers" in December
% of 1991. Nonetheless, the core principle of the lottery, six numbers between 1 and 49 has not changed once in almost 70 years and thus provides a consistent
% basis for our work.

\todo{Is there a dataset for a rigged lottery?} \href{https://notebook.community/JesseScott/Lotto649/lotto}{This} looks sketchy as hell.

However even 70 years of Lotto numbers is not sufficient to produce enough data for the diehard tests. These require 10 to 12 MiB of random bits, which is
substantially more than 28.4 KiB of Lotto numbers. Therefore we downloaded other lottery datasets and combined them together. In total, our dataset reached
more than 750 000 numbers drawn in 18 different lotteries. These majority of these lotteries come from various english-speaking countries such as USA,
Australia or UK because we have been looking for them with English search queries. Most of the complement is formed by other european nations like Italy,
Czech Republic or Germany.

\todo{Merge datasets?}

The numbers are drawn individually, but their order within a single lottery draw does not matter - but maybe it does for some of our tests?

\todo{Investigate this}


\section{Methods}
\label{sec:methods}
In the first part we test whether the drawn numbers come from a uniform distribution $\mathcal{U}(1, 50)$. To investigate this,
we use Pearson's $\chi^2$ test and the Kolmogorov-Smirnov test.

For more information about the Diehard battery of tests we refer the reader to the original paper~\cite{currentRNG}.


\section{Results}
\label{sec:results}
\subsection{Hypothesis test results}

The $\chi^2$ goodness-of-fit test for the distribution of the German Lotto distribution 
of $d$ yielded a p-value of $0.4988592$. This is not significant at $\alpha = 0.05$ or 
any other common significance level.

This means that the German Lotto distribution of $d$ looks like a very typical
distribution for a fairly conducted lottery. We have no reason to suspect
human tampering from this result.

\section{Diehard tests}
\label{sec:diehard}
Fairness entails more than the question whether are lottery numbers from the expected distribution. For instance, the Kolmogorov-Smirnov
test used in the first part of this paper does not concern itself with the order the numbers are drawn. However if we saw a lottery whose
numbers were always drawn in a descending sequence, for example, we would become suspicious.

Thus a more comprehensive test is clearly required to establish a more detailed answer to our question. We approach this problem by reformulating
lottery as a process producing a stream of (supposedly) random numbers, which themselves are simply bit sequences. Under this formulation,
we can deploy standard statistical tests developed for testing random number generators: we have a file of one and zero bits and wish to investigate
if its bits are correlated, repeating with a period or other quantities undesirable for randomness.

A number of these test suites has been developed over time. Donald Knuth presented an initial set of empirical tests in the second volume of his computer science bible
The Art of Computer Programming in 1969. Many general cryptography textbooks such as \href{http://www.cacr.math.uwaterloo.ca/hac/}{Handbook of Applied Cryptography} 
or \href{http://www.wisdom.weizmann.ac.il/~oded/foc-vol1.html}{Foundations of Cryptography} contain multiple tests of their own. The american National Institute 
of Standards \& Technology has published a \href{https://nvlpubs.nist.gov/nistpubs/legacy/sp/nistspecialpublication800-22r1a.pdf}{guideline} discussing this matter too.

We decided to use the Diehard battery of tests, which was developed by the american statistician George Marsaglia in the nineties. While this package used
to be quite popular in its day, it has been superceded today by other suites, including its derivatives such as Dieharder or TestU01. In comparison with the alternatives,
the Diehard tests \href{https://crypto.stackexchange.com/questions/90076/how-to-compute-the-dataset-size-required-by-dieharder-tests}{consume a lot less data}, making
its use feasible to limited data cases such as our project.

Nonetheless even less data greedy test still requires quite a lot of data. We are thus limited in which tests we can run, see~\ref{fig:dataset} for our dataset size.

\begin{figure}
    \centering
    \includegraphics[width=\textwidth]{diehard_requirements.pdf}
    \caption{Amount of data required to run tests at default settings.}
    \label{fig:requirements}
\end{figure}

\subsection{Data augmentation}

Diehard battery expects a random stream of zero and one bits as input. To satisfy this, we transform our lottery numbers distributed in 1-n range by first
subtracting 1 and taking 5 lowest bits from every number for $n \in \langle 32, 64)$ or 6 for $n \in \langle 64, 128)$.

This transformation doesn't however solve all problems. The winning numbers are sometimes provided in sorted, ascending order in a single draw. We discuss
this in further sections. Furthermore, the numbers are drawn without replacement. We argue that this is an acceptable deviation. \todo{A bit of combinatorics,
combinations with replacement as approximation for combinations without replacement}.

\subsection{Results}

\begin{figure}
    \centering
    \includegraphics[width=\textwidth]{performance.pdf}
    \caption{Score at various Diehard tests.}
    \label{fig:performance}
\end{figure}

\begin{figure}
    \centering
    \includegraphics[width=\textwidth]{pvalue_distribution.pdf}
    \caption{Randomness quality as measured by Diehard. Gray line is perfect randomness.}
    \label{fig:pvalue_distribution}
\end{figure}

\begin{enumerate}
    \item About Diehard.
    \item The following part is divided into several section. In section 1, we discuss data gathering, processing and general creation of input files for Diehard. Special
    attention is given to the problem of attaining input file of sufficient size and the asymetric requirements of individual Diehard tests.
    \item Part 2 highlights result on Diehard suite, interprets them and compares them against commonly used PRNGs.
    \item Section 3 clarifies the shortcomings of above approach.
\end{enumerate}


\section{Conclusion}
\label{sec:conclusion}
\begin{enumerate}
    \item Lottery draws without replacement
    \item Some datasets are sorted in draw order
    \item Reminder: too few numbers for comfort
    \item The Diehard tests are flawed (\href{https://crypto.stackexchange.com/a/397}{Linear Feedback Shift Registers})
\end{enumerate}

\bibliography{sources.bib}
\bibliographystyle{unsrt}

\end{document}
